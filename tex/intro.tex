\section{Introduction}
\label{intro}
The NuCypher network ~\cite{egorov2017nucypher} is a decentralized key management 
system, encryption, and access control service. It uses proxy re-encryption, namely, the 
Umbral  scheme~\cite{umbral2018}, to delegate access to encrypted 
documents through a public network. This network is composed of a set of semi-trusted 
re-encryption servers, or Ursulas, that implement access control policies created by data 
owners, or Alices. Alice supplies each Ursula with re-encryption keys allowing Ursula to transform 
a ciphertext encrypted under Alice's public key into a ciphertext decryptable by the delegatee's, or Bob's,  
public key. This enables the latter to decrypt the data without revealing anything about the 
decryption keys or the raw data to the intermediate servers.


Joining the NuCypher network is governed by consensus rules defining the service setup 
and the monetary incentives paid to provide the service. In order to join the 
system, an Ursula needs to stake an amount of NU tokens for a specific period 
and will be released when the pre-specified staking period is over. Alice chooses an  
Ursula to hold an access control policy with a probability proportional to the stake this Ursula 
pledged in the system. As such, the stake value influences the amount of service load Ursula 
may receive. This stake is also used to punish Ursula financially, by revoking part 
of it, when cheating is detected. 


Alice joins the network by creating a policy to be implemented by a 
set of Ursulas. This policy specifies the re-encryption key fragments each Ursula will
need to answer requests coming from Bob(s), as well as the duration of the policy, which 
is basically the timeframe curing which Bob is authorized to access Alice's data.  Furthermore, 
Alice will lock an amount of Ether into the policy to be used to pay Ursulas for the re-encryption 
service. As noted, Alice is not exposed to the NU token. All what she needs is an Ethereum 
wallet, being aware of the NuCypher network architecture to select Ursulas, and knowing the 
NuCypher rules of preparing access control policies.


Bob, on the other hand, does not deal with currency in the system. It interacts with the storage 
network to retrieve the encrypted data, and with Ursulas in the NuCypher network to ask for the  
re-encryption service.


\subsection{Monetary Incentives in the NuCypher Network}
Ursulas are rewarded for the re-encryption service by using two sources; 
the first is the service fees collected from the policy or data 
owner Alice, while the second is an inflation rewards, i.e., newly minted 
NU tokens, coming from the NuCypher network. The latter will be provided during the 
early stage of the network operation to encourage 
adoption. It is expected that these rewards will disappear as the token cap is 
reached, much similar to other cryptocurrencies out there.


Currently, the inflation rewards are computed for each Ursula based on the size of their stake, and their confirmation that they are online, once per period. Confirming the status of being online is done by having each Ursula call a 
simple function in one of the NuCypher contracts to signal that she is around. 

As for fees, Ursula collects the Ether locked by Alice during the policy's duration in proportion to the number of periods up to that point in which Ursula has confirmed their online status. 


\subsection{Problems with Current Reward Computation Approach}
The current approach of computing Ursula's rewards suffers from two 
main problems: 
\david{It's important to clarify here if we're talking about inflation rewards, service fees, or both.}
\begin{itemize}
\setlength{\itemsep}{0pt}
\item The rewards computation is agnostic to the number of 
requests Ursula serves. Thus an active Ursula who may serve a huge number of 
requests, and an idle one who does not receive any requests (or a 
malicious Ursula that does not answer the requests she receives from Bob), will 
both collect the same rewards if they hold identical policies.
\david{We should make a distinction between being idle because there are no requests or being off-line. The former is a legitimate situation driven by factors outside Ursula's control, while the latter is a punishable action.}
\item Confirming being online is flawed in the sense that this function does 
not require any input or proofs from Ursula on providing the service. Ursula 
can stay offline, not answering any Bob, and be online just to call the confirm activity 
function. This allows Ursula to collect both inflation rewards 
and fees without doing anything, and even worse, it will be able to get her stake back once it unlocks (policies being managed by Ursula will have expired by this point). 
\end{itemize}



\subsection{Proposed Solutions}
We need a provably secure mechanism to be used by Ursula to report the 
number of distinct Bob requests it handled during a specific period, and use this number 
to compute the fees to be collected from Alice. Thus, we call this confirm 
service activity, as opposed to confirm activity that refers to solely being available 
online.


Although we can use the number of served requests to compute the inflation 
rewards as well, this will encourage Ursula to collude with Bob to issue large 
number of requests. If inflation rewards were to be distributed based on the earning of fees (which is desirable, as it would link node subsidization more closely with the network service), this might encourage Ursula to deploy its own Alice(s) 
and Bob(s) to increase the demand, and thus, the cut of the inflation rewards.


In this document, we introduce several solutions for confirming service activity. 
These solutions differ in the trust, efficiency, interactivity, and resource 
requirements, in addition to their security guarantees. We present each of these 
solution, where one of them is still a work in progress, discuss the aforementioned 
trade-offs and requirements of each of them, with the goal of selecting one of these 
solutions to be adopted in the NuCypher network.


\subsection{Roadmap}
The rest of this document is organized as follows. Section~\ref{threat-network-model} 
outlines the network and threat models adopted by the proposed solutions. 
Section~\ref{prelim} provides an overview of the underlying cryptographic primitives and 
protocols, while the proposed solutions are presented 
in Section~\ref{solutions} with their security and efficiency aspects analyzed in 
Section~\ref{analysis}. Lastly, directions of future work on the confirm activity issue 
are outlined in Section~\ref{future-work}.


