\section{Introduction}
\label{intro}
The NuCypher network~\cite{egorov2017nucypher} is a decentralized key management 
system, encryption, and access control service. It uses proxy re-encryption, namely, the 
Umbral scheme~\cite{umbral2018}, to delegate access to encrypted 
documents through a public network. This network is composed of a set of semi-trusted 
re-encryption servers, or Ursulas, that implement access control policies created by data 
owners, or Alices. Alice supplies each Ursula with re-encryption keys allowing Ursula to transform 
ciphertexts encrypted under Alice's public key into ciphertexts encrypted under the delegatee's, or Bob's,
public key. This enables the latter to decrypt the ciphertext without revealing anything about the
decryption keys or the raw data to the intermediate servers.


Joining the NuCypher network is governed by consensus rules defining the service setup
and the monetary incentives paid to provide the service. In order to join the 
system, Ursula needs to stake an amount of NU tokens for a specific period
which will be released when the pre-specified staking period is over. Alice chooses an
Ursula to hold an access control policy with a probability proportional to the stake this Ursula 
pledged in the system. Therefore, the stake value influences the amount of service load an Ursula
may receive. This stake is also used to punish Ursula financially, by revoking part 
of it, if she cheats and this cheating is detected. 


Alice joins the network by creating a policy to be implemented by a 
set of Ursulas. This policy specifies the re-encryption key fragments each Ursula will
need to answer requests coming from Bob(s), as well as the duration of the policy, which 
is basically the timeframe during which Bob is authorized to access Alice's data.  Furthermore,
Alice will lock an amount of Ether into the policy to be used to pay Ursulas for the re-encryption 
service. As noted, Alice is not exposed to the NU token. All that she needs is an Ethereum
wallet, awareness of the NuCypher network architecture to select Ursulas, and knowledge of the
NuCypher rules of preparing access control policies.


Bob, on the other hand, does not deal with currency in the system. It interacts with the storage 
network to retrieve the encrypted data, and with Ursulas in the NuCypher network to request the
re-encryption service.


\subsection{Monetary Incentives in the NuCypher Network}
Ursulas are rewarded for the re-encryption service by using two sources; 
the first is the service fees collected from the policy  
owner Alice, while the second is inflation rewards, i.e., newly minted
NU tokens, coming from the NuCypher network. The latter will be provided during the 
early stage of the network operation to encourage 
adoption. It is expected that these rewards will disappear as the token cap is 
reached, similar to other cryptocurrencies i the space.


Currently, the inflation rewards for each round or period are computed for each Ursula based on the size of her stake, and the confirmation that she is online. Confirming the status of being online is done by having each Ursula call a 
simple function in one of the NuCypher contracts to signal that she is online.

As for fees, Ursula collects part of the Ether locked by Alice during the policy's duration in proportion to the number of periods up to that point in which Ursula has confirmed her online status. 


\subsection{Problems with Current Reward Computation Approach}
The current approach of computing Ursula's inflation rewards and service fees suffers from two
main problems:
\begin{itemize}
\setlength{\itemsep}{0pt}
\item The rewards computation (both fees and inflation rewards) is agnostic to the number of 
requests Ursula serves. Thus, an active Ursula who may serve a huge number of 
requests, and one who does not receive any requests (or a
malicious/lazy Ursula that does not answer the requests she receives from Bob), will 
both collect the same amount of rewards if they hold identical policies.

\item Confirming being online is flawed in the sense that this function does 
not require any input or proofs from Ursula on providing the service. Ursula 
can stay offline and not respond to Bobs' requests, and simply come online merely to call the confirm activity
function. This allows Ursula to collect both inflation rewards 
and fees without doing any work. Even worse, Ursula will be able to get her stake back once she unlocks (i.e., when policies being managed by Ursula expire). 
\end{itemize}



\subsection{Problem Statement}
The confirm activity problem is defined differently based on the stage of the 
NuCypher network. In early network operation, when inflation rewards are 
still distributed, we are concerned with the availability of Ursulas and being willing to 
serve all requests coming from Bob. In other words, regardless of the number of 
requests served, the rewards value will be computed based on the period during which Ursula 
is online and well-behaving. While in later stages, when inflation rewards disappear 
and only fees exist, confirm activity will be tied to providing the re-encryption 
service in the sense that the payment will be computed based on the amount of 
service provided. Thus, Ursula needs to confirm its activity by proving that she served 
a given number of distinct re-encryption requests during a given period. To make the distinction 
clear, we refer to the first as confirm availability, and for the second as confirm 
service activity.


In this document, we introduce several solutions for both forms of confirming activity. 
These solutions differ in the trust, efficiency, interactivity, and resource 
requirements, in addition to their security guarantees. We divide the document into 
two parts, the first is about confirm availability, while the second is about confirm 
service activity. We begin with outlining the adopted threat model for each problem 
definition, as well as the network model Then we present our solutions for each 
of these problems, discuss the aforementioned
trade-offs and requirements of each of them, with the goal of selecting one of these 
solutions to be adopted in each stage of the NuCypher network operation.


\subsection{Threat and Network Models}
\label{threat-network-model}
This section describes a modification for the network model of NuCypher 
required by the proposed solutions, in 
addition to the threat model adopted in this work.


\subsubsection{Network Model.}
In order for the proposed solutions to work, we need to ensure freshness, 
integrity, and authenticity of the requests (or service compliants) issued by Bob. This can be achieved 
by requiring Bob to sign each request it issues and to include a timestamp, or 
sequence number, in each request to ensure freshness. The keypair used for the 
signature should be separate from the keypair Bob uses for proxy encryption in the 
system. 


Until now we assume that Alice pays for the service by using the Ether she locks 
in the policy contract. Other arrangement may emerge in the system like having 
Bob pay for each requests he issues. Such an arrangement and its implication on the 
confirm service activity issue will be studied once it becomes part of the NuCypher 
network protocol.


\subsubsection{Threat Model.}
We do not place trust in any party and we assume that all participants are 
self-interested. This means that a party may decide to 
follow the protocol or deviate from it, either on its own or by colluding 
with other attackers, and such decision is solely based on what maximizes the financial profits
of this party. Thus, if it is profitable, an Ursula may, for example, spin out  
its own Alice and/or Bob or collude with Alice. We do not consider collusion 
between Bob and Ursula as a practical threat. This means that cases of Bob pretending  
that he got service from Ursula (while no service is delivered) is not an issue. We do not 
suspect this will be of any importance and there is no motivation to do it given that 
the work Ursula does for re-encryption is minimal.


The above non-collusion assumption does not affect the solutions proposed to handle 
the confirm service activity discussed in Section~\ref{confirm-service-actiivty}. This 
means that these solutions provide higher 
security guarantees and address the issue f potential collusion between Bob and Ursula. 
It could be the case that relaxing the requirement of 
addressing this collusion case allows deploying more efficient solutions. 
In order to make an educated decision of the plausibility of this threat, we need 
more data about the system operation and the behavior of the participants 
once the inflation rewards disappears in the system. As such, we leave this 
issue until later when fees become the only source of rewards for Ursulas. 


We also assume that when sampling a subset of Ursulas in the network, 
at least one of them is honest. This assumption can be achieved by 
having a global assumption regarding the lower bound of the number of 
honest Ursulas with respect to the total number of Ursulas in the system. 
(Or it can be achieved by deploying a special entity like an external verifier, 
that changes on a periodic basis, or one by the NuCypher company, that is
trusted to faithfully participate as a member of each samples set. This 
verifier does not provide any other services 
in the system.)


Other than the above, we have usual assumptions like dealing with 
computationally bounded adversaries 
that cannot break secure cryptographic primitives with non-negligible probability. 
(We may need to work in the random oracle model, but this depends on the 
security requirements of the proposed solutions.)



