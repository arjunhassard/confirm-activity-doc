\section{Introduction}
\label{intro}
The NuCypher network~\cite{egorov2017nucypher} is a decentralized key management 
system, encryption, and access control service. It uses proxy re-encryption, namely, the 
Umbral scheme~\cite{umbral2018}, to delegate access to encrypted 
documents through a public network. This network is composed of a set of semi-trusted 
re-encryption servers, or Ursulas, that implement access control policies created by data 
owners, or Alices. Alice supplies each Ursula with re-encryption keys allowing Ursula to transform 
ciphertexts encrypted under Alice's public key into ciphertexts encrypted under the delegatee's, or Bob's,
public key. This enables the latter to decrypt the data without revealing anything about the
decryption keys or the raw data to the intermediate servers.


Joining the NuCypher network is governed by consensus rules defining the service setup
and the monetary incentives paid to provide the service. In order to join the 
system, Ursula needs to stake an amount of NU tokens for a specific period
which will be released when the pre-specified staking period is over. Alice chooses an
Ursula to hold an access control policy with a probability proportional to the stake this Ursula 
pledged in the system. Therefore, the stake value influences the amount of service load an Ursula
may receive. This stake is also used to punish Ursula financially, by revoking part 
of it, if it cheats and this cheating is detected. 


Alice joins the network by creating a policy to be implemented by a 
set of Ursulas. This policy specifies the re-encryption key fragments each Ursula will
need to answer requests coming from Bob(s), as well as the duration of the policy, which 
is basically the timeframe during which Bob is authorized to access Alice's data.  Furthermore,
Alice will lock an amount of Ether into the policy to be used to pay Ursulas for the re-encryption 
service. As noted, Alice is not exposed to the NU token. All that she needs is an Ethereum
wallet, awareness of the NuCypher network architecture to select Ursulas, and knowledge of the
NuCypher rules of preparing access control policies.


Bob, on the other hand, does not deal with currency in the system. It interacts with the storage 
network to retrieve the encrypted data, and with Ursulas in the NuCypher network to request the
re-encryption service.


\subsection{Monetary Incentives in the NuCypher Network}
Ursulas are rewarded for the re-encryption service by using two sources; 
the first is the service fees collected from the policy  
owner Alice, while the second is inflation rewards, i.e., newly minted
NU tokens, coming from the NuCypher network. The latter will be provided during the 
early stage of the network operation to encourage 
adoption. It is expected that these rewards will disappear as the token cap is 
reached, similar to other cryptocurrencies out there.


Currently, the inflation rewards are computed for each Ursula based on the size of their stake, and their confirmation that they are online, once per period. Confirming the status of being online is done by having each Ursula call a 
simple function in one of the NuCypher contracts to signal that she is online.

As for fees, Ursula collects part of the Ether locked by Alice during the policy's duration in proportion to the number of periods up to that point in which Ursula has confirmed its online status. 


\subsection{Problems with Current Reward Computation Approach}
The current approach of computing Ursula's inflation rewards and service fees suffers from two
main problems:
\begin{itemize}
\setlength{\itemsep}{0pt}
\item The rewards computation (both fees and inflation rewards) is agnostic to the number of 
requests Ursula serves. Thus an active Ursula who may serve a huge number of 
requests, and one who does not receive any requests (or a
malicious/lazy Ursula that does not answer the requests she receives from Bob), will 
both collect the same amount of rewards if they hold identical policies.

\item Confirming being online is flawed in the sense that this function does 
not require any input or proofs from Ursula on providing the service. Ursula 
can stay offline and not respond to Bobs' requests, and simply come online merely to call the confirm activity
function. This allows Ursula to collect both inflation rewards 
and fees without doing any work. Even worse, Ursula will be able to get her stake back once she unlocks (i.e., when policies being managed by Ursula expire). 
\end{itemize}



\subsection{Overview}
The confirm activity problem is defined differently based on the stage of the 
NuCypher network. In early network operation, when inflation rewards are 
still distributed, we are concerned with the availability of Ursulas and being willing to 
serve all requests coming from Bob. In other words, regardless of the number of 
requests served, the rewards value will be computed based on the period during which Ursula 
is online and well-behaving. While in later stages, when inflation rewards disappear 
and only fees exist, confirm activity will be tied to providing the re-encryption 
service in the sense that the payment will be computed based on the amount of 
service provided. Thus, Ursula needs to confirm its activity by proving that she served 
a given number of distinct re-encryption requests during a give period. To make the distinction 
clear, we refer to the first as confirm availability, and for the second as confirm 
service activity.


In terms of the underlying threat model, for both problem definitions we do not consider collusion 
between Bob and Ursula as a practical threat. This means that cases of Bob pretending  
that he got service from Ursula (while no service is delivered) is not an issue (we do not 
suspect this will be of any importance and there is no motivation to do it given that 
the work Ursula does for re-encryption is minimal).


In this document, we introduce several solutions for both forms of confirming activity. 
These solutions differ in the trust, efficiency, interactivity, and resource 
requirements, in addition to their security guarantees. We divide the document into 
two parts, the first is about confirm availability, while the second is about confirm 
service activity. We present each of our solutions, discuss the aforementioned
trade-offs and requirements of each of them, with the goal of selecting one of these 
solutions to be adopted in each operational stage of the NuCypher network.




