\section{Confirm Availability}
In this section, we present a solution for the issue of confirming availability of 
Ursulas. As mentioned before, in the early stage of the network launch we only 
require Ursulas to be online and responsive to any reencryption 
requests issued by Bob. 


We begin with the design details of our solution, after which we discuss its 
resilience to any potential security attacks and its efficiency in terms of implementation 
requirements and compatibility with the current status of the NuCypher network.


\subsection{Optimistic Challenge-based Approach}
The proposed solution is centered around two important concepts: requiring minimal 
changes to the current network implementation, and minimizing the additional overhead. 


Recall that in the NuCypher network, Bob receives a map from Alice containing 
the set of Ursulas that has the key fragments needed to implement the access control policy. 
When issuing a reencryption request, Bob uses this map to reach each of these Ursulas to obtain 
the service. This communication is one-to-one meaning that no other Ursulas (or any 
NuCypher network participant) mediates the communication between Bob and the target  
Ursula. This in turn means that no one can attest to whether Ursula has responded if Bob 
complains that he did nit hear back.


The main idea is to add a mediator, or potentially a witness, in the service process (if needed) to monitor 
a specific target Ursula and challenge its responsiveness. Hence, it is an optimistic protocol invoked 
only when Bob complains about not receiving the service. In detail, for each round (where a 
round is the time needed to mine a block on the blockchain) a set of Ursulas is selected at 
random. We call these gateway Ursulas. Bob contacts a target Ursula for its request as usual. 
If this Ursula does not respond, Bob complains to a gateway that this target Ursula 
did not respond. At this point, the gateway will act as an intermediary and forwards 
Bob's request to the target Ursula on behalf of Bob and waits for the answer. If the target 
Ursula does not answer, this will trigger other gateways to perform the same process, and if the target 
Ursula is still unresponsive, these gateways will collectively sign a witness against 
this Ursula and publish it (see Section~\ref{cosi} for an overview of collective signing). 
Once this witness is verified by the NuCypher smart contract responsible of service policies, 
part of the target Ursula's stake will be slashed as a punishment.


\paragraph{Gateway Ursulas election.} 


\paragraph{Selecting gateway Ursulas to handle a Bob's complaint.}


\paragraph{Witness verification.}


\paragraph{Quantifying the financial punishment or slashing value.}


\paragraph{Operating only during challenge phase.}


\paragraph{DoS attacks.}


\subsection{Security Analysis}


\subsection{Performance Analysis}