\subsection{Threat and Network Models}
\label{threat-network-model}
This section describes a modification for the network model of NuCypher 
required by the proposed confirm service activity solutions, in 
addition to the threat model we adopt in this document.


\subsubsection{Network Model}
In order for the proposed solutions to work, we need to ensure freshness, 
integrity, and authenticity of the requests issued by Bob. This can be achieved 
by requiring Bob to sign each request it issues and to include a timestamp, or 
sequence number, in each request to ensure freshness. The keypair used for the 
signature should be separate from the keypair Bob uses for proxy encryption in the 
system. 


Until now we assume that Alice pays for the service by using the Ether she locks 
in the policy contract. Other arrangement may emerge in the system like having 
Bob pay for each requests he issues. Such arrangement and its implication on the 
confirm service activity issue will be studied once it becomes part of the NuCypher 
network protocol.


\subsubsection{Threat Model}
We do not place trust in any party and we assume that all participants are 
self-interested. This means that a party may 
follow the protocol or deviate from it, either on its own or by colluding 
with other attackers, and decisions are solely based on what maximizes the financial profits
of this party. Thus, Ursula may collude with Bob or Alice (by spinning 
its own Alice and/or Bob for example) to achieve her financial goals. 
The latter is assumed in the current proposed solutions (PCD, CoSi, and 
challenge/open based schemes). This means that solutions provide higher 
security guarantees. It could be the case that relaxing the requirement of 
accounting for Bob/Ursula collusion allows deploying more efficient solutions. 
In order to make an educated decision of the plausibility of this threat, we need 
more data about the system operation and the behavior of the participants 
once the inflation rewards disappears in the system. As such, we leave this 
issue until later when fees become the only source of rewards for Ursulas. 


We also assume that when sampling a subset of Ursulas in the network, 
at least one of them is honest. This assumption can be achieved by 
having a global assumption regarding the lower bound of the number of 
honest Ursulas with respect to the total number of Ursulas in the system. 
(Or it can be achieved by deploying a special entity like an external verifier, 
that changes on a periodic basis, or one by the NuCypher company, that is
trusted to faithfully participate in the confirm 
service activity protocol. This verifier does not provide any other services 
in the system.)


Other than the above, we have usual assumptions like dealing with 
computationally bounded adversaries 
that cannot break secure cryptographic primitives with non-negligible probability. 
(We may need to work in the random oracle model, but this depends on the 
security requirements of proposed solutions.)
