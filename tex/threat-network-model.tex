\section{Threat and Network Models}
\label{threat-network-model}
This section describes a modification for the network model of NuCypher 
required by the proposed confirm service activity solutions, in 
addition to the threat model we adopt in this document.


\subsection{Network Model}
In order for the proposed solutions to work, we need to ensure freshness, 
integrity, and authenticity of the requests issued by Bob. This can be achieved 
by requiring Bob to sign each request it issues and to include a timestamp, or 
sequence number, in each request to ensure freshness.

\david{NuCrypher currently supports this by allowing Bob to include a blockhash,
although it's not being used for the moment.}

The following approach can be used here; given that Umbral already requires Bob to 
have a key pair, this pair can be used for signing and verifying the signed requests. 
This also means that Ursula should be aware the secret key can be used for 
signing the requests and the public 
key should be known to Ursula to verify the signatures. 

\david{Reusing the same keypair for encryption and signing may introduce some key management problems. Is it possible to find a solution that supports separate keypairs?}

\subsection{Threat Model}
We do not place trust in any party and we assume that all participants are 
self-interested. This means that a party may 
follow the protocol or deviate from it, either on its own or by colluding 
with other attackers, is solely based on what maximizes the financial profits 
of this party. As such, Ursula may collude with Bob or Alice (by spinning 
its own Alice and/or Bob for example) to achieve her financial goals. 


We also assume that when sampling a subset of Ursulas in the network, 
at least one of them is honest. This assumption can be achieved by 
having a global assumption regarding the lower bound of the number of 
honest Ursulas with respect to the total number of Ursulas in the system. 
(Or it can be achieved by deploying a special entity like an external verifier, 
that changes on periodic basis, or one by the NuCypher company, that is 
trusted to faithfully participate in the confirm 
service activity protocol. This verifier does not provide any other services 
in the system.)


Other than the above, we have usual assumptions like dealing with 
computationally bounded adversaries 
that cannot break secure cryptographic primitives with non-negligible probability. 
(We may need to work in the random oracle model, but this depends on the 
security requirements of proposed solutions.)
